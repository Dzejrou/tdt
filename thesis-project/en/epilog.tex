\chapter*{Conclusion and future work}
\addcontentsline{toc}{chapter}{Conclusion and future work}

In conclusion, let us verify that  our game conforms all of our goals listed in Section~1.4. The main goal of this thesis was to design
and implement a 3D dungeon management game using the design elements \textbf{(E1)}~--~\textbf{(E6)} that we defined in Section~1.1:

\begin{enumerate}[label=\textbf{(E\arabic*)}]
    \item Resource management -- our game contains two resources that the player manages from which one, called gold, must be mined by
        the player's minions and is used to build new buildings and unlock new research nodes and one, called mana, regenerates by
        itself and is used for spell casting.
    \item Dungeon building -- the players of our game can order their minions to tear down walls and can place new buildings on
        unobstructed places in the game world. The game contains 18 different building with various purposes, such as minion
        spawning and defending against enemies.
    \item Minion commanding -- the players of our game can use 4 different spells to command their minions. These spells are
        \emph{mine}, which can be used to command minions to mine gold, \emph{attack}, which can be used to command minions to attack
        a particular enemy, \emph{reposition}, which can be used to command minions to move to a specific area, and \emph{return\_gold},
        which can be used to pull gold reserves that our minions currently hold back to the gold vaults.
    \item Combat -- our game contains a wave system that cause the player's dungeon to be periodically under attack from enemies that wish
        to destroy the dungeon throne. The player's minions defend the dungeon from these enemies.
    \item Player participation in combat -- the game contains various different combat spells that the player can use to directly affect
        the outcome of a battle.
    \item Research -- the game contains a research interface with 42 research nodes that can be used to unlock new spells and buildings as
        well as to apply one time bonuses to the player and their minions.
\end{enumerate}

As we can see in the list above, our game satisfies every of the design elements and as such we believe that it can be called a dungeon
management game.

Besides the main goal, we had several other goals that our game was supposed to satisfy. The first of these goals, \textbf{(G1)}, required
us to create a full competetive product and not just a game prototype. While the graphical fidelity of our game might get judged as low,
we believe that from programming and content points of view we managed to satisfy this goal as the game features full single player
experience with various minions, spells and buildings and enemies that pose a challenge for our players and create the possibility of a
loss. Additionally, our goal \textbf{(G1.1)} required our game to be performant and to run above the minimum acceptable framerate,
defined in Section~1.3 as 25-30 frames per second, which we believe the game has also satisfied, because the framerate we saw during
tests on various machines -- both older and newer -- almost never dropped below 100.

Another of our goals, \textbf{(G2)}, required us to allow our players to modify our game. These modification were supposed to be able
to modify and create new entities and spells, to alter the game's progression system and to create custom levels.
We believe we satisfied this goals because all of the content in the game -- that is, minions, enemies, building, spells, enemy waves and
others -- are implemented using the game's own modding tools and as such our players possess the same ability to modify the game that
we do. Additionally, because of the format chosen for level serialization our players can create custom levels that can not only
modify the data of the game, but also its mechanics.

The last of our goals, \textbf{(G3)}, required us to provide our players with means to create easily installable mods. Our game allows
both the creation of mods that are loaded by the game on its startup that only require the player to add them to the game's
configurational scripts and the creation of mods that are a part of a custom level, which only require placement within the
\texttt{saves} directory and loading of the custom level. Because of this, we believe that we also this last goal.

To summarize, we believe that we satisfied all of the goals that we have set for ourselves in Section~1.4.

\subsubsection{Future work}

While we believe our game to be feature complete, we think that there is a room for improvement of the game.
Following list contains -- in no particular order -- our ideas for possible future extensions of the game.

\begin{itemize}
    \item An editor that can be used to create modifications for our game as mentioned in Section~2.1. The main reason for not creating
        such editor for our game was that it would require the implementation of a graphical scripting language that would allow our
        players to define new behavior and abilities of the different entities in the game.
    \item Icons for spells and buildings, which would help easier orientation in the spell casting and building tool bars. In the current
        state of the game, these are represented as text and as such might not be as intuitive to use as they could be if they had
        graphical icons that would indicate their purpose or effect.
    \item Tooltips for spells, buildings and research nodes with descriptions and resource costs. These would further simplify the user
        experience of our game and are not present in the game because of their lack in the graphical skin used by our GUI. We believe
        that either implementing them into the graphical skin or creating a new skin that supports button tooltips would help our players
        to play game in a more intuitive manner.
    \item Better models, materials and textures for entities. The game contains only very simple colored models or models that are distributed
        with Ogre3D. The addition of better graphics would, in our opinion, increase the pleasantness of the game to the human eye.
    \item Animations. These were originally planned to be implemented in the game but were not implemented because of time constraints
        and a lack of animated models. Similarly to the previous point, we believe that this feature would increase the competetivness
        of our game on the video game market.
    \item Special component that would be able to represent components defined in Lua. This feature would further increase the modifiability
        of the game. Additionally, the option to implement new systems that correspond to these components in Lua would be also beneficial.
\end{itemize}

In addition to these possible extensions, we believe that further expansion of the modding API would help greatly to increase the
modifiability of our game as the current modding API -- while fairly large -- was designed mainly to suit the needs of the game and as such
may limit mods that would try to radically diverge from the play style of the game.
