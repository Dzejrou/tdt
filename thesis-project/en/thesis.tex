%%% The main file. It contains definitions of basic parameters and includes all other parts.

%% Settings for single-side (simplex) printing
% Margins: left 40mm, right 25mm, top and bottom 25mm
% (but beware, LaTeX adds 1in implicitly)
\documentclass[12pt,a4paper]{report}
\setlength\textwidth{145mm}
\setlength\textheight{247mm}
\setlength\oddsidemargin{15mm}
\setlength\evensidemargin{15mm}
\setlength\topmargin{0mm}
\setlength\headsep{0mm}
\setlength\headheight{0mm}
% \openright makes the following text appear on a right-hand page
\let\openright=\clearpage

%% Settings for two-sided (duplex) printing
% \documentclass[12pt,a4paper,twoside,openright]{report}
% \setlength\textwidth{145mm}
% \setlength\textheight{247mm}
% \setlength\oddsidemargin{14.2mm}
% \setlength\evensidemargin{0mm}
% \setlength\topmargin{0mm}
% \setlength\headsep{0mm}
% \setlength\headheight{0mm}
% \let\openright=\cleardoublepage

%% Character encoding: usually latin2, cp1250 or utf8:
\usepackage[utf8]{inputenc}

%% Further useful packages (included in most LaTeX distributions)
\usepackage{amsmath}        % extensions for typesetting of math
\usepackage{amsfonts}       % math fonts
\usepackage{amsthm}         % theorems, definitions, etc.
\usepackage{bbding}         % various symbols (squares, asterisks, scissors, ...)
\usepackage{bm}             % boldface symbols (\bm)
\usepackage{graphicx}       % embedding of pictures
\usepackage{fancyvrb}       % improved verbatim environment
\usepackage{natbib}         % citation style AUTHOR (YEAR), or AUTHOR [NUMBER]
\usepackage[nottoc]{tocbibind} % makes sure that bibliography and the lists
			    % of figures/tables are included in the table
			    % of contents
\usepackage{dcolumn}        % improved alignment of table columns
\usepackage{booktabs}       % improved horizontal lines in tables
\usepackage{paralist}       % improved enumerate and itemize
\usepackage[usenames]{xcolor}  % typesetting in color
\usepackage{url}
\usepackage{enumerate}
\usepackage[shortlabels]{enumitem}
\usepackage{booktabs, caption, fixltx2e}
\usepackage[flushleft]{threeparttable}
\usepackage[scaled=.95]{inconsolata}
\usepackage{listings}
\usepackage{tikz, tikz-qtree}
\usepackage{float}
\usepackage{svg}
\usepackage{xspace}

\newcommand*{\incon}[1]{{\fontfamily{fi4}\selectfont #1}}
\def\cpp{{C\nolinebreak[4]\hspace{-.05em}\raisebox{.4ex}{\tiny\bf ++\xspace}}}

\lstset{language=[5.2]Lua, basicstyle=\ttfamily, columns=fullflexible, showstringspaces=false, commentstyle=\color{gray}}

% Listing float:
\newfloat{listing}{tbhp}{lst}
\floatname{listing}{Listing}

%%% Basic information on the thesis

% Thesis title in English (exactly as in the formal assignment)
\def\ThesisTitle{The Dungeon Throne: A 3D Dungeon Management Game}

% Author of the thesis
\def\ThesisAuthor{Jaroslav Jindrák}

% Year when the thesis is submitted
\def\YearSubmitted{2016}

% Name of the department or institute, where the work was officially assigned
% (according to the Organizational Structure of MFF UK in English,
% or a full name of a department outside MFF)
\def\Department{Department of Distributed and Dependable Systems}

% Is it a department (katedra), or an institute (ústav)?
\def\DeptType{Department}

% Thesis supervisor: name, surname and titles
\def\Supervisor{Mgr. Pavel Ježek, Ph.D.}

% Supervisor's department (again according to Organizational structure of MFF)
\def\SupervisorsDepartment{Department of Distributed and Dependable Systems}

% Study programme and specialization
\def\StudyProgramme{Computer~Science}
\def\StudyBranch{Programming and Software Systems}

% An optional dedication: you can thank whomever you wish (your supervisor,
% consultant, a person who lent the software, etc.)
\def\Dedication{%
I would like to thank my supervisor, Pavel Ježek, for his patience and advice that he has provided to me.
Without his guidance this thesis would hardly reach the level of quality I believe it to have.

I would also like to thank my parents, who supported me throughout my work on this thesis and without whom I would
never be able to dedicate the time needed to finnish this project.
}

% Abstract (recommended length around 80-200 words; this is not a copy of your thesis assignment!)
\def\Abstract{%
    Dungeon Keeper is an iconic and genre-defining game that places the player in the role of a dungeon master that
    has to protect their dungeon against groups of heroic enemies that try to steal the treasures located in the
    dungeon. However, it does not provide any tools that would allow user created modifications and thus the replayability
    of the game might not reach its full potential.

    In this thesis, we examine the dungeon management genre and the aspects of game modifiability and use the knowledge gained
    from this examination to design a spiritual successor to Dungeon Keeper that offers tools that allow its players
    to create and distribute modifications that can be applied to the game.

    We then investigate different tools and approaches that can be used for game development and use those we find
    to be suitable to implement our game. The game we created is true to the play style of the original Dungeon Keeper
    and provides an API that can be used in the Lua programming language to create modifications of the game's elements.
}

% 3 to 5 keywords (recommended), each enclosed in curly braces
\def\Keywords{%
    {dungeon management game}, {Lua scripting}, {Ogre}, {CEGUI}
}

%% The hyperref package for clickable links in PDF and also for storing
%% metadata to PDF (including the table of contents).
\usepackage[pdftex,unicode]{hyperref}   % Must follow all other packages
\hypersetup{breaklinks=true}
\hypersetup{pdftitle={\ThesisTitle}}
\hypersetup{pdfauthor={\ThesisAuthor}}
\hypersetup{pdfkeywords=\Keywords}
\hypersetup{urlcolor=blue}

% Definitions of macros (see description inside)
\include{macros}

% Title page and various mandatory informational pages
\begin{document}
\include{title}

%%% A page with automatically generated table of contents of the bachelor thesis

\tableofcontents

%%% Each chapter is kept in a separate file
\chapter{Introduction}

Until the release of Dungeon Keeper\footnote{Bullfrog Productions, 1997} most well known fantasy 
video games have allowed the player to play as various heroic characters, raiding dungeons filled
with evil forces in order to aquire treasures and fame.
In Dungeon Keeper, however, we join the opposite faction and try to defend our own dungeon
(along with all the treasures hidden in it) from endless hordes of heroes trying to pillage our domain.
Although we can still play the original Dungeon Keeper today, we cannot change its data or game mechanics
in any easy way so this thesis aims to recreate and modify the original game and extend it to have an easy to 
use programming interface that will allow such modifications.

\section{Dungeon Managment Genre}

Dungeon Keeper was the first game released in the dungeon management~(DM) genre and since our game is going to be based on
Dungeon Keeper, we should design it with the elements of its genre in mind. Since the definition of this genre has to our
knowledge never been formally documented by the creators of Dungeon Keeper, we are going to create a list of basic elements
of the genre based on the gameplay of the original game.

In Dungeon Keeper, the player's main goal is to build and protect his own base, called the dungeon. They do so by commanding
their underlings (often called minions), whom they can command to mine gold, which is a resource they use in order to build new rooms
and cast spells (in the game's sequel\footnote{Dungeon Keeper 2, Bullfrog Productions, 1999}, mana was added into the game
as a secondary resource used for spell casting), which could be researched as the game progressed. They would then use creatures
spawned in their buildings as well as their own magic powers to fight intruders in order to protect their dungeon. 
From this brief gameplay summary, we can create list of the most basic design elements, which can be found in dungeon management games:

\begin{enumerate}[label=\textbf{(E\arabic*)}]
    \item Resource management
    \item Dungeon building
    \item Minion commanding
    \item Combat
    \item Player participation in combat
    \item Research
\end{enumerate}

Now that we have created a list of the genre's basic design elements, let us have a look at how they were implemented in games from the
Dungeon Keeper series.

\subsubsection{Resource management}

In the original Dungeon Keeper, the player used gold as their primary resource. They would have it mined by their minions and use
it to build new rooms and cast spells. While having a single resource for everything may bring simplicity to the game, it also means
that once the player runs out of gold, they wont be able to directly participate in combat due to their inability to cast spells.
It is for this reason that we are going to use the resource model of Dungeon Keeper 2 and have a separate resource -- mana -- that
will be used to cast spells.

\bigskip
Ask PJ: Should the word mana be itallic in the same way mods and modding tools are? This is its second appearance in the text, but the
first one wasn't "definitioney" and a general player should know the concept.

\subsubsection{Dungeon building}

The term dungeon building in Dungeon Keeper refers to tearing down walls in order to create new rooms which the player's minions 
then claim and the player can place new buildings in. These buildings can then act as gold storage, spawn new creatures or be used
as traps that negatively affect attacking enemies. Since this is a central theme in Dungeon Keeper (and dungeon management games
in general), we should try to implement our building model to resemble this.

\subsubsection{Minion commanding}

The term minion commanding stands for giving your minion tasks such as to move somewhere, attack an enemy or tear down a wall. In
Dungeon Keeper, most of these commands were implemented as spells -- although mining was done by simply selecting blocks, which
could unfortunately cause minions to destroy walls the player accidentally selected. To providea unified interface to minion tasks
and to avoid the accidental block destruction, all commands the player can give will be in our game implemented as spells.


\subsubsection{Combat}

In Dungeon Keeper, the player's dungeon is under frequent attacks by enemy heroes. These enemies attack the dungeon 
in groups with a delay between each two attacks, in a similar way to tower defense games (e.g. Orcs must die!~\cite{OMD} 
and Dungeon Defenders~\cite{DungeonDefenders}), where theay are often referred to as \emph{waves}. Each wave of enemies
can consist of different types of heroes and the delay between ways can differ, too. Since the players of the original
Dungeon Keeper are alredy familiar with this wave system and it also caters to players of tower defense games, we will be
implementing it in our game.

\subsubsection{Player participation in combat}

During the fights with enemy heroes, the player in Dungeon Keeper can cast various spells to affect the outcome of the battle.
These spells can have various forms from spawning creatures, damaging enemies and healing minions to destroying walls and throwing
meteors. If we want to create a similar spell system, we need to support these different types of spells, e.g. targeted spells that
affect a single entity, positional spells that can summon a creature or global spells, that simply have an effect on the game once
they are cast.

\subsubsection{Research}

In the original Dungeon Keeper, researching new spells and buildings was done through a special type of room, called the library,
which can be seen in Figure~\ref{dk-lib}. There,
various minions could study and achieve new advancements for the player after a certain amount of research points -- which was different
for different spells and buildings -- was gathered. But this design decision brought a negative element into the game, because once the
library was destroyed by the heroes, the player was unavailable to perform additional research and could even use the ability to cast
some spells. This is why we are going to use a different approach, called the research tree (or sometimes technology tree), first used
in the real-time stragety game called Sid Meyer's Civilization\footnote{Developed by MPS Labs, 1991}~\cite{ResearchTrees}. This approach 
separates the research from the events happening in the game and allows the player to research new advancements for a resource (e.g. gold),
with each new technology, building, creature or spell being able to have prerequisites.

\begin{figure}[h]
    \centering
    \includegraphics[width=10cm]{../img/library.jpg}
    \caption{Warlocks researching in the library.
             \\Source: \href{http://vignette2.wikia.nocookie.net/dungeonkeeper/images/9/98/Library.jpg/revision/latest?cb=20120808211437}{http://dungeonkeeper.wikia.com}}
    \label{dk-lib}
\end{figure}

\subsubsection{Conclusion}

We have now investigated the main design elements of Dungeon Keeper (and dungeon management games in general) and how we are going 
to implement them in our game. The last thing we have to realise is that since we are trying to cater to the players of the original
game, the resulting product of our work should be a full game and not just a prototype. This means that the game should offer full 
singleplayer experience, with relatively intelligent enemies and the ability to not only win the game, but to also lose. It also means
that the game has to be performant, achieving atleast the minimum acceptable framerate, defined as 25-30 frames per second by 
Shiratuddin, Kitchens and Fletcher~\cite{AcceptableFPS}, on both new and older hardware.

\section{Modifiability in Games}

One of our basic goals is for our game to be modifiable, which means to provide tools -- often called \emph{modding tools} -- to our players
that will allow them to create modifications -- often called \emph{mods} -- that other players can install and which can change or
add elements to the game.

This can increase the replay value of the game as after finishing it, more missions, characters, game mechanics, abilities, items
or even game modes can be easily downloaded and installed from internet. Since we want our game to be modifiable, we should allow our players
to change most (if not all) of the game's data, artificial intelligence and abilities of all minions, creatures and enemies and also give
them the ability to edit levels the game, which would allow them to create new maps.

An important topic we need to decide on is which parts of the game we will allow the players to modify. 
An example of an easily modifiable game is Minecraft~\cite{Minecraft}, a 3D sandbox game 
in which the player has to survive in a procedurally generated world, which can be seen in Figure~\ref{minecraft}. 
The player builds a house, crafts tools, mines for materials and
fights off enemies during the night. These elements of the game provide a wide array of things that mods can change. For example to smelt
ores into ingots, the player can use a furnace powered by coal, but a mod called Industrial Craft~2~\cite{IndustrialCraft} added machines
into the game that, among other uses, could smelt ingots using electricity, another concept introduced by the mod. The player can then
create power plants of different types -- ranging from solar to nuclear -- to produce electricity that they can then use to power their
machines. With this mod, the player can create setups that allow him to automate tasks such as smelting, cooking or mining. An example of
such setup can be seen in Figure~\ref{ic-solar}.

\begin{figure}[h]
    \centering
    \includegraphics[width=10cm]{../img/minecraft.jpg}
    \caption{A procedurally generated world in Minecraft.
             \\Source: \href{http://i.neoseeker.com/p/Games/PC/Simulation/City/minecraft\_image\_zx2AU2n6bZho0lz.jpg}{http://www.neoseeker.com}}
    \label{minecraft}
\end{figure}

\begin{figure}[h]
    \centering
    \includegraphics[width=10cm]{../img/ic-solar.pdf}
    \caption{A simple setup powering multiple machines including an electrical furnace using solar panels.
             \\Source: \href{http://static.planetminecraft.com/files/resource\_media/screenshot/1246/javaw-2012-11-12-20-51-09-46\_4122139.jpg}{http://www.planetminecraft.com}}
    \label{ic-solar}
\end{figure}

Another mod, called ComputerCraft~\cite{ComputerCraft}, added fully functional computers into the game. 
These computers can be used to create password protected
doors, write programs and games in Lua or even connect to internet services such as IRC\footnote{Internet Relay Chat, an open communication 
protocol.}. An example of such computer in minecraft can be seen in Figure~\ref{computer-craft}. This mod also added robots into the game, 
that could be programmed using Lua scripts to automate tasks such as mining, fighting or woodcutting.

\begin{figure}[h]
    \centering
    \includegraphics[width=10cm]{../img/ComputerCraft2.png}
    \caption{A computer in Minecraft playing the ASCII version of the movie Star Wars: A New Hope.
             \\Source: \href{http://minecraft-modding.de/wp-content/uploads/2015/06/ComputerCraft2.png}{http://www.minecraft-modding.de}}
    \label{computer-craft}
\end{figure}

In Minecraft, the player has access to simple tools, e.g. a pickaxe or a sword, but with mods, the game can be ehnanced with advanced 
technologies or even magic (e.g. Thaumcraft or MineMagicka). This can create new ways the game is played and possibly extend its
replayability.

Aside from adding items and creatures, entire new games can be created within a modifiable game. For example Minecraft's command block,
which can execute commands in the game's developer console -- e.g. teleport the player to a certain position or give him items --
when interacted with, was used by a player of the game to create~\cite{FutureOfMinecraft} a custom map that copied the game 
Team Fortress~2~\cite{TF2}. Team Fortress~2 is a 3D first person shooter, in which two player teams fight against each other in various
game modes, e.g. capture the flag, deathmatch or point control. Each player could choose from a wide selection of character classes, 
each with its
own attributes and weapons. The command block was used for example to equip players based on the character class they choose -- 
which can be done by choosing between different command blocks each being assigned to a single class and a console command that 
adds an item to a player -- or controlling a point, for which a command block can check if a player stands on a point long enough
to capture it.

\bigskip
Ask PJ: Should I explain the terms CTF, DM and point controol? Most players should already know them :/
\\Also, when I mentioned Thaumcraft and MineMagicka, the only official resources are their forum posts which have long urls
and can change them over time (updating patch version in the post title alters the url -.-), should I refer to them anyway?
\bigskip

To allow the creation of custom maps in our game, we should store our levels
in a format that will allow later modifications, including actual changes to the game's elements.

In conclusion, we can see that the ability to modify a game can help said game to grow even when its development has stopped
or is focused in different areas (e.g. security, stability). Since we want to give this ability to the players of
our game, our modding tools should allow them to change most of its data, including but not limited to:

\begin{itemize}
    \item Minions and enemies -- e.g. changing attributes such as health and damage, displayed models, behavior and spells they
        cast.
    \item Buildings -- e.g. changing their size, models, types of creatures that they spawn and, in the case of traps, their interaction with
        enemies.
    \item Spells -- fully changing the effect of a spell, e.g. from simple damage dealing to spawning a meteor shower.
    \item Goals of the game -- changing requirements for winning the game or the reasons for a loss.
\end{itemize}

Besides changing data of entities -- e.g. creatures, buildings or spells -- the players should also be able to create new types of these
entities.

The game on its own should also be fully featured, offering enough of these entities
on its own so the players do not need mods to actually play the game. Additionally, since our game, like Dungeon Keeper,
will have scripted waves of enemies attacking the player's dungeon, the also should allow our players to alter the wave 
composition -- that is, which types of enemies compose the different groups attacking the player's dungeon --  and delays between
the waves. Last, but not least, we must not forget that players do not necessarily have to be (and often are not)
programmers, so our game should provide an easy way to install these modifications.

\section{Thesis Goals}

The main goal of this thesis is to design and implement a modifiable 3D dungeon management game using the design elements \textbf{(E1)}~--~\textbf{(E6)}.

In addition to the main goal, the game should complete the following list of goals:

\begin{enumerate}[label=\textbf{(G\arabic*)}]
    \item The game has to be a full competetive product, not a prototype.
        \begin{enumerate}[label=\textbf{(G1.\arabic*)}]
            \item It has to be performant, achieving high framerate even on low end computers.
            \item It has to offer full single player experience, with scripted enemies and a chance to both win and lose.
            \item It has to contain a variety of entities, spells and buildings even without mods.
        \end{enumerate}
    \item The game has to be highly modifiable, providing an easy to use modding interface for players.
        \begin{enumerate}[label=\textbf{(G2.\arabic*)}]
            \item The mod creators must be able to create new entities, spells and buildings and to change most of the game's data.
            \item They must also be able to alter the game progression by defining enemies that spawn and delays between them.
            \item The game should also support the creation of custom levels.
        \end{enumerate}
    \item The mods for the game have to be easily installable even by players without any programming knowledge. 
\end{enumerate}

\chapter{Problem Analysis}

% TODO some intro?

\section{Programming Language}

% TODO
\begin{itemize}
    \item talk about the need of fast language that can be used to create
        3D games with the ability to be scripted using an embedded language
        (or itself)
    \item talk about the benefit of runtime code execution (testing)
    \item talk about the necessity of scripting without compilation
\end{itemize}

\subsection{C++}

% TODO
\begin{itemize}
    \item industry standard
    \item powerful and fast
    \item modifiability can be added through embedded languages
    \item lots of materials
\end{itemize}

\subsection{Lua}

% TODO
\begin{itemize}
    \item slower than C++, faster than other scripting languages
    \item can be scripted by itself
    \item there are no 3D rendering libraries
    \item high portability
\end{itemize}

\subsection{Java}

% TODO
\begin{itemize}
    \item Minecraft proved that it allows highly modifiable game
        development
    \item it also shown that the performance can be a big problem
    \item high portability
\end{itemize}

\subsection{C\#}

% TODO
\begin{itemize}
    \item explain that the only reason this was not picked over C++
        is that I did not have much experience with it when I started
        working on the game
\end{itemize}

\section{Engine Design}

% TODO
\begin{itemize}
    \item maybe mention that the goal to make an engine and that's why UE or Unity
        weren't used
    \item say that since the game is to be modifiable and extensible, run time
        entity creation is a must (both for modding and testing)
    \item might be good mention the choice before explaining why the different
        options were/were not chosen so that the reader knows
\end{itemize}

\subsection{Inheritance based (wording?)}

% TODO
\begin{itemize}
    \item explain what is meant by the title (classic OOP design)
    \item this kind of engine would also have to be planned a lot (to avoid
        inheritance hierarchy hell)
    \item ?say that it's not as performant as the other two?
\end{itemize}

\subsection{Component based (wording?)}

% TODO
\begin{itemize}
    \item define the Component design pattern (with references to
         the gamedesginpatterns.smth book)
    \item mention that this is very similar to the component based implementation
         both Unity and UE4 use
    \item mention its relationship with both ECS and the inheritance based design
\end{itemize}

\subsection{Entity Component System}

% TODO
\begin{itemize}
    \item define the ECS design pattern
    \item mention how well it supports run time entity creation and modification
    \item mention ease of development this model creates
    \item mention ease of entity access (through ID)
    \item talk about the difference between Component and ECS patterns
    \item ?mention cache friendliness? (pro: well, it exists, con: not used that
        much in this game)
\end{itemize}

\section{Scripting Language}

TODO: Add more possible scripting languages, possibly C++ using dlls?
% TODO
\begin{itemize}
    \item should be fast, easily embedded into C++ and also
        easily used by modders
\end{itemize}

\subsection{Custom Language}

% TODO
\begin{itemize}
    \item challenge, experience, can be created to suit the project
    \item slow development, big task
    \item probably slow and buggy
\end{itemize}

\subsection{Lua}

% TODO
\begin{itemize}
    \item industry standard
    \item maybe talk about some of the different games using it
    \item ease of embedding into C++
    \item no need to create anything, just connect it to C++
\end{itemize}

\subsubsection{Bindings}

% TODO
\begin{itemize}
    \item talk about most bindings being either dead, obsolete or very
        archaic
    \item those that are up to date are mostly focused on OOP in Lua,
        which isn't needed in this project
    \item creating new binding just for this project is easy, fast and
        the resulting binding will only contain the features that are needed
\end{itemize}

\section{Libraries}

% TODO
\begin{itemize}
    \item just mention something about reinventing the wheel and
        the areas for which 3rd party libraries were used
\end{itemize}

\subsection{3D Rendering}

TODO: Merge OpenGL and DirectX together as low level API and add more high level libs.
% TODO
\begin{itemize}
    \item almost any game will have to use some library for this as
        going without would take years
    \item ??I only though about these in the beggining, should I add
        others just for comparison??
    \item mention that portability of the rendering library is much more
        important than for example the GUI library, since it's integrated
        deep into the engine while the GUI is not
\end{itemize}

\subsubsection{OpenGL}

% TODO
\begin{itemize}
    \item talk about what it is
    \item would allow later Linux port
    \item too low level, would slow the development process
\end{itemize}

\subsubsection{DirectX}

% TODO
\begin{itemize}
    \item talk about what it is
    \item similar problem to OpenGL (too low level)
    \item would prohibit eventual port to Linux
\end{itemize}

\subsubsection{Ogre3D}

% TODO
\begin{itemize}
    \item talk about what it is
    \item can wrap both OpenGL and DirectX
    \item higher level, lots of stuff already implemented
    \item describe basic usage of the library (entities, meshes, scene graph, ...) 
\end{itemize}

\subsection{Graphical User Interface}

% TODO
\begin{itemize}
    \item main requirement is Ogre compatibility
    \item theoretically does not have to be portable as it can easily be changed
        (as it's not integrated into the engine)
    \item other requirements: graphical editor, ease of use, high level, all the
        widgets needed (buttons, scrollbars, editboxes, labels, ...)

\end{itemize}

\subsubsection{Ogre Overlay}

% TODO
\begin{itemize}
    \item talk about what it is
    \item too simple, would require for the widgets to be created during
        the game development process which would take a lot of time
\end{itemize}

\subsubsection{CEGUI}

% TODO
\begin{itemize}
    \item talk about what it is
    \item used to be bundled with Ogre
    \item easy to use, clean interface, good design
    \item ??? genrally advised by the Ogre community
    \item ??? used for Torchlight which proved what can be done with it
    \item graphical editor and existing skins
    \item ??I only though about these in the beggining, should I add
        others just for comparison??
\end{itemize}

\section{Game Design}

% TODO
\begin{itemize}
    \item go over which features of the dungeon management genre will be used and why
    \item possibly talk about how they could be done? or leave this to implementation?
\end{itemize}

\section{Algorithms}

% TODO
\begin{itemize}
    \item dunno about this section, not many known or big
        algorithms used in the game
\end{itemize}

\subsection{Pathfinding}

% TODO
\begin{itemize}
    \item talk about the problem of pathfinding
    \item talk about node graph vs vertex graph (or hows it called)???
    \item after mentioning node graph, give reason for grid form
	    and 8 neighbours (bcuz of the view)
    \item also talk about the inclusion of portals in pathfinding
	    and problems that have risen with it
\end{itemize}

\subsubsection{Breadth-first Search}

% TODO
\begin{itemize}
    \item just mention what it is and why it's not good
	    (uniform cost of graph edges)
\end{itemize}

\subsubsection{Dijkstra}

% TODO
\begin{itemize}
    \item talk about what it is and why it's not the best alternative
    \item explain that it might be worth using when performance is
	    not critical and we absolutely need the best path
\end{itemize}

\subsubsection{A*}

% TODO
\begin{itemize}
	\item talk about what it is (possibly also define heuristic as it should
		be seen for the first time now)
	\item mention that it's probably the most used one in games (provide examples?)
	\item ??mention how it goes well with e.g. portals??
	\item maybe go into detail about the default heuristic (PORTAL\_HEURISTIC) which
		allows portal pathfinding (include previous iterations, portal component,
		problem with portal chains and possible fix)
\end{itemize}

\subsubsection{Solution ??}

% TODO
\begin{itemize}
    \item maybe this should be in the A* part as it's the only one actually implemented?
    \item describe the generic pathfinding system implemented in the game, switchable
	    algorithms, heuristics and path types (define path type),
	    algorithm and path type interface ...
\end{itemize}

\subsection{Level Generation ??}

% TODO
\begin{itemize}
    \item talk about how complex level generation isn't really needed due
	    to the nature of the game (you dig your own hallways so only need
	    gold distribution)
    \item mention that similar generic system to the pathfinding one is used
    \item ??probably go through the algorithm in pseudocode?? or explain it only??
	    if so, probably go through the part with the gold only and just mention
	    the rest??
\end{itemize}

\section{Serialization}

% TODO
\begin{itemize}
    \item talk a bit about game state serialization in general
    \item mention that since the game should be extensible, manual save game
	    editing without the need to write big editors would be nice
	    for testing and possibly the creation of simple mod maps
	    (and the save editor - if ever created - would be easy to make)
\end{itemize}

\subsection{Binary}

% TODO
\begin{itemize}
    \item mention that it's the most used
	    (maybe give examples like UE4 save system etc.)
    \item it's also the most compact as it just meshes data together in binary
    \item lots of libraries (like boost)
    \item editing would require complex editor and does not allow code execution in the
	    save file so additional save file functionality (like changing the wave
	    system table, executing custom commands) would need to be made in C++
	    in a generic way, which would be bad for mods
\end{itemize}

\subsection{XML}

% TODO
\begin{itemize}
    \item mention losts of xml libraries
    \item talk about how it allows easy data change, which in a data driven game like this
	    one means quite a lot
    \item but explain that it's bad because it disallows code execution
\end{itemize}

\subsection{Lua}

% TODO
\begin{itemize}
    \item since the game needs an extensive modding api with all sorts of setters/getters,
	    all that's needed is to write serialization of components into a sequence of
	    the API calls
    \item easy to implement, load is basically executing a script
    \item easy to modify using a text editor
    \item !!allows code execution to create special levels
    \item say that it's limitation is level size, e.g. 256x256 save file can be big (200MB) and
	    very slow to load (10-20sec), but on the allowed level sizes (10-64) the load
	    is almost instant and the resulting files small (16x16 is around 650KB)
\end{itemize}

\section{Spells}

Here just compare the original idea of having the spell system in C++ and the current version
of defining spells fully in lua and C++ just call callbacks basically.

\chapter{Developer's Documentation}

In this chapter, we will examine the structure of our game and the implementation of its
various modules. In Figure~\ref{engine-layout}, we can see the main modules of our engine
and their relationships -- an arrow pointing from module A to module B means that A uses B.

\begin{figure}[h]
    \centering
    \fontfamily{fi4}\selectfont
    \def\svgwidth{\columnwidth}
    \input{engine.pdf_tex}
    \caption{Core of the game's engine.}
    \label{engine-layout}
\end{figure}

The game uses four different libraries -- Ogre3D for graphics, OIS\footnote{Bundled with Ogre3D.} for user input, CEGUI for
GUI and Lua for scripting. Ogre3D and OIS are used directly by the engine, but CEGUI and Lua -- besides initialization in the
\texttt{Game} class -- use special interface modules that handle their API. The GUI module, which acts as an interface
for CEGUI, comprises the \texttt{GUI} class and various different classes that represent different GUI widgets. Lua uses different
classes for the different directions of access. When the engine -- written in C++ -- accesses Lua, it uses the \texttt{lpp::Script} class,
which acts as a wrapper facade over the Lua C API. When Lua accesses parts of the engine, it uses the \texttt{LuaInterface} class,
which binds functions written in C++ to Lua.

The update logic of the game is performed by systems which, with the exception of \texttt{EntitySystem}, are equal in functionality.
\texttt{EntitySystem} stands out because it, besides updating the game, aslo acts as a component manager and provides components
to the rest of the systems.

In the following sections we will examine these main modules of the engine as well as various other, which mostly serve as tools
to the main modules.

\section{Game}

The \texttt{Game} class is the central part of our engine. In its constructor, every module of the engine -- that needs initialization --
is initialized. This class also serves as the main connection between the Ogre3D and OIS libraries and our engine.It does so by inheriting
four listener classes from these libraries -- \texttt{Ogre::FrameListener}, \texttt{Ogre::WindowEventListener}, \texttt{OIS::KeyListener}
and \texttt{OIS::MouseListener}.

\begin{listing}
    \centering
    \begin{lstlisting}[language=C++]
// Ogre3D
bool frameRenderingQueued(const Ogre::FrameEvent&);
void windowResized(Ogre::RenderWindow*);
void windowClosed(Ogre::RenderWindow*)

// OIS
bool keyPressed(const OIS::KeyEvent&);
bool keyReleased(const OIS::KeyEvent&);
bool mouseMoved(const OIS::MouseEvent&);
bool mousePressed(const OIS::MouseEvent&,
                  OIS::MouseButtonID);
bool mouseReleased(const OIS::MouseEvent&,
                   OIS::MouseButtonID);
    \end{lstlisting}
    \caption{Virtual functions overriden in the \texttt{Game} class.}
    \label{ois-ogre-callbacks}
\end{listing}

Communication between the game and these two libraries is done, besides using their API, by overriding virtual functions of these classes
which act as event handlers. We can see these functions in Listing~\ref{ois-ogre-callbacks}. While names of most of these functions are
self-explanatory -- and their documentation can be found in the Ogre3D and OIS documentations -- we will examine the functionality of
\texttt{frameRenderingQueued} in a bit more depth.

\texttt{Ogre::FrameListener} provides three virtual functions -- \texttt{frameStarted}, which is called when a frame
is about to begin rendering, \texttt{frameEnded}, which is called when a frame has just been rendered, and \texttt{frameRenderingQueued},
which gets called when all rendering commands have been issued and are queued for the GPU to process. In our \texttt{Game} class, we override
only this last function -- although others can be overriden for different actions as well -- because while the GPU processes the current
frame, the CPU can be used for different tasks -- such as our update loop. This means that any changes that happen inside this function
will be rendered on the next frame, but in our game this delay is not noticeable as the player does not directly control any one entity.
We use this function to perform the main functionality of our game -- performing game update by calling the update functions of all
systems that are in the game.

Besides handling input and window events, this class contains several auxiliary functions related to initialization, level creation and
state changing. To see more information about these functions, refer to their documenting comments in the header file or in the
generated documentation.

\section{Components}

Component is a data aggregate that describes a specific characteristic of an entity that contains the component. In our engine, these
components are represented as simple structures with a name in the form of \texttt{<Characteristic>Component}. 

\begin{listing}
    \centering
    \begin{lstlisting}[language=C++]
struct MovementComponent
{
    static constexpr int type = 4;

    MovementComponent(tdt::real speed = 0.f)
        : speed_modifier{speed}, original_speed{speed}
    { /* DUMMY BODY */ }

    tdt::real speed_modifier;
    tdt::real original_speed;
}
    \end{lstlisting}
    \caption{Simplified representation of the MovementComponent structure.}
    \label{component-ex}
\end{listing}

In Listing~\ref{component-ex}, we can see an example of a component -- the MovementComponent, which describes the speed of an entity.
Every component has a static integer field representing its type, which is used for communication between Lua and C++ because Lua does not
understand the notion of templates, which we use for the different functions that manipulate with components, and as such it needs
to specify what kind of component should be used when it calls a C++ function.

There is only one other requirement for components, which is that they always need to have a parameterless constructor
\footnote{A constructor that has default values for all its parameters will suffice.}. This is required because
we often create components by calling a C++ function in Lua and then set its fields, which would be impossible without a constructor that
doesn't need any parameters. Also note that because of this approach, any strings passed to any parameterized constructor are considered
to be passed from Lua and as such are \textbf{moved} because the originals are supposed to be destroyed by Lua once the creation is complete.

Besides the two requirements, components can contain any data required by the entity characteristic they describe. In the examle, the
MovementComponent contains two fields of type \texttt{tdt::real}\footnote{One of two numeric types used in the engine, the other being
tdt::uint.}
-- speed\_modifier, which indicates the speed of an entity, and original\_speed, which is used
to restore the speed of an entity that has been affected by a slowing effect. Note that, to conform our
ECS representation, components should not contain functions and any logic should be done through systems.

\section{Systems}

A system is a class that performs a part of the games update. The game provides a common abstract parent class \texttt{System}, which is
used for easy iteration over all systems during the update of the game. In general, a system should operate over one or more types of
components, but the creation of systems that do not use components is also possible.

The update function of every system is called once per frame, but the individual systems often have inner time periods between updates.

\subsection{EntitySystem}

\texttt{EntitySystem} is the main system as, besides performing part of the game's update, it acts as a component database. This means that
it stores all component containers, which are represented as \texttt{std::map<entity\_id, component>}. Besides storing the components,
it also provides a set of function templates used for component manipulation.

\begin{listing}
    \centering
    \begin{lstlisting}[language=C++]
template<typename COMP>
bool has_component(tdt::uint id);

template<typename COMP>
COMP* get_component(tdt::uint id);

template<typename COMP>
void add_component(tdt::uint id);

template<typename COMP>
void delete_component(tdt::uint id);
    \end{lstlisting}
    \caption{Examples of the templates using for component manipulation.}
    \label{es-comp-manip-ex}
\end{listing}

Some of these function templates can be seen in Listing~\ref{es-comp-manip-ex}. They can be used to test if an entity has a component,
to retrieve a component
that belongs to an entity, add a new component -- using the parameterless constructor -- to an entity and remove a component from an entity,
respectively. In addition to these templates, each of them has a secondary non-templated version which takes an integer as a second parameter,
which corresponds to the type of the component -- these functions are then used from Lua.
To prevent code repetition, these functions use an array of pointers to the templated versions in which the pointer at any given index points
to the instances of these templated functions that have their type field equal to the index number.

In addition to these public templates, one more important private template is defined in \texttt{EntitySystem} -- \texttt{load\_component}.
This function template accepts the identifier of an entity along with a name of a Lua table, which it will then use to load a component
with fields specified by the provided table.

Besides component storage and manipulation, this class also serves as a system. During its update, it deletes all components and entities
that were scheduled for deletion by the function \texttt{delete\_component}. The reason for this delayed delete is that if we deleted
components from their containers immediately in the delete call, we might invalidate iterators as the call may very well happend during
an iteration over a component container.

\subsection{CombatSystem}

The \texttt{CombatSystem} class performs the update of basic combat between entities -- that is, melee
\footnote{By melee we refer to close range combat as is common in many games.} and ranged combat excluding spells.
During its update, it checks the ability to attack of all entities that currently have an active combat target. This includes checking
if the target is in sight and in range. If an entity can attack, the system either performs applies its damage to its target if the
attack is of the melee type or creates a new projectile if the attack is of type ranged.
After updating the combat state of all entities that are currently fighting, the system updates the movement of all projectiles.

This system also performs a special kind of pathfinding which is used to find a path for an entity that is trying to escape from an attacker.
This pathfinding, unlike the general pathfinding that finds a path to the target, uses a queue and is performed once per frame. The reason for
this is that this pathfinding does not need to be performed immediately, while the general pathfinding is often used to check for the
existence of a path and as such needs to be finnished right after its start so that the return value can be used.

In addition to performing part of the game's update, \texttt{CombatSystem} provides two important types of functions -- querying for closest
entity that satisfies a condition, applying an effect to all entities in range that satisfy a condition.

\subsubsection{Conditions and entity querying}

Conditions are functors -- that is, structures that overload the function call operator -- which are used to test if an entity satisfies
a certain requirement. \texttt{CombatSystem} provides a function template that can be used to find the closest entity that satisfies
a condition.

\begin{listing}
    \centering
    \begin{lstlisting}[language=C++]
template<typename CONT, typename COND>
tdt::uint get_closest_entity(tdt::uint, COND&, bool);
    \end{lstlisting}
    \caption{Signature of the entity query function template.}
    \label{cond-ex}
\end{listing}

In Listing~\ref{cond-ex}, we can see the signature of this function template. Its template parameters are \texttt{CONT}, which specifies
the type of component container the function will be querying over, and \texttt{COND}, which specifies the condition functor. The function
takes the identifier of the entity that performs the query, instance of the condition functor and a boolean value determining if the
target has to be in sight and returns the identifier of the closest entity that satisfies the condition.

\subsubsection{Effects and their application}

In addition to entity querying, \texttt{CombatSystem} provides a function template that allows us to apply an effect to all entities
that satisfy a condition and are withing range. An effect, similarly to a condition, is a functor which is used to affect the entity
it is used on.

\begin{listing}
    \centering
    \begin{lstlisting}[language=C++]
template<typename CONT, typename COND, typename EFFECT>
void apply_effect_to_entities_in_range(tdt::uint, COND&,
                                       EFFECT&, tdt::real);
    \end{lstlisting}
    \caption{Signature of the effect applying function template.}
    \label{effect-ex}
\end{listing}

The signature of this function template can be seen in Listing~\ref{cond-ex}. Its template parameters are the same as the ones used
in entity querying with the addition of the parameter \texttt{EFFECT}, which specifies the effect functor. The function takes the
identifier of the entity that applies the effect, the condition the targets need to satisfy, the effect that will be applied to the
targets and the maximal range the targets can be from the applying entity.

\subsection{EventSystem}

\texttt{EventSystem} manages event handling as part of the game update. To do this, it uses two different components --
\texttt{EventComponent}, which represents an event, and \texttt{EventHandlerComponent}, which represents the ability of an entity
to handle events.

During its update, this system iterates over all events in the game and, if they are active -- finds a suitable entity that will
handle the event. Events can be of two types, either targeted or area events. An event is targeted if it has a valid entity
identifier in its \texttt{handler} field, otherwise it is an area event.

To request an entity to handle an event, the system checks if the entity can handle it by testing the event type in the entity's
\texttt{possible\_events} bitset field. Every \texttt{EventHandlerComponent} has a string field \texttt{handler}, which specifies
which Lua table contains the event handling function called \texttt{handle\_event}. The system then calls this event handling
function and passes the identifier of the handling entity and the identifier of the event to it, which causes the entity
to handle the event.

To handle a targeted event, the system simply finds the event's handler entity and requests event handling, but to handle an area event,
the system has to iterate over different event handlers that are within radius of the event, which is specified in the
\texttt{EventComponent}'s field \texttt{radius} and increases on every update call until the event is destroyed. The system requests
all of the suitable entities to handle the event until any one entity returns \texttt{true}, in which case it destroys the
\texttt{EventComponent} of the event. The reason for destroying the component and not the entity that represents the event is that
an entity that causes the event often gains the component so that the lifetime of the event is bound to the entity -- e.g. a falling
meteor can have an \texttt{EventComponent} of type \texttt{METEOR\_FALLING} and when the meteor hits the ground, both the meteor and
the event get destroyed since they are one entity.

\subsection{GridSystem}

% TODO
\begin{itemize}
    \item describe the update logic (+freed/unfreed and Grid relation)
    \item talk about grid graphics used for debugging?
    \item explain structure placement
    \item describe alignment checks
\end{itemize}

\subsubsection{Grid}

% TODO
\begin{itemize}
    \item explain its purpose and implementation
    \item mention that it only provides a set of grid related operations
        on a set of IDs it's provided (and assumes those are grid)
    \item mention random node placement and entity distribution
    \item mention free node list for ease (and speed) of access
    \item explain graph creation and linking
    \item explain general usage
\end{itemize}

\subsection{TaskSystem}

% TODO
\begin{itemize}
    \item describe the update logic
    \item talk about busy state, processing tasks and checking if task is complete
    \item probably only forward reference the Lua part? or talk about it right here? !ask PJ!
\end{itemize}

\subsection{WaveSystem}

% TODO
\begin{itemize}
    \item describe states and maybe mention why the State design patter wasn't used (too simple)
    \item describe the update logic of the different states (-> countdown, spawning, chilling)
    \item talk about the spawning mechanism, blueprint vector, spawn nodes
    \item talk about wave entity monitoring and wave ending
    \item talk about the wave table
    \item talk about the wstart and wend callbacks
    \item talk about endless mode
    \item possibly explain how to write a wave? or add that to the scripting part?
\end{itemize}

\subsection{Miscellaneous Systems}

% TODO
\begin{itemize}
    \item here explain in alphabetical order the remaining systems
    \item say why these are not as important (that is, not important to be described) as the others
\end{itemize}

\subsubsection{AISystem}

% TODO
\begin{itemize}
    \item describe the update logic
    \item mention update forcing (in relation to the update period mentioned earlier)
\end{itemize}

\subsubsection{GraphicsSystem}

% TODO
\begin{itemize}
    \item say how it's supposed to maintain all manual graphics logic but atm only does explosions
        (since it's the only manual graphics object)
    \item describe the update logic
\end{itemize}

\subsubsection{HealthSystem}

% TODO
\begin{itemize}
    \item describe the update logic
    \item mention configurable regeneration period
\end{itemize}

\subsubsection{InputSystem}

% TODO
\begin{itemize}
    \item describe the update logic
    \item explain initial purpose and why it's not used atm
\end{itemize}

\subsubsection{ManaSpellSystem}

% TODO
\begin{itemize}
    \item describe the update logic (both mana regen and entity spell casting)
    \item possibly talk about the relationship between player spells and entity spells
\end{itemize}

\subsubsection{MovementSystem}

% TODO
\begin{itemize}
    \item describe the update logic
    \item talk aboud move and checked\_move (and can\_move\_to) and why checked is not used
        but can be from lua (and was intended for 1st person mode)
\end{itemize}

\subsubsection{ProductionSystem}

% TODO
\begin{itemize}
    \item describe the update logic
    \item explain how products are spawned and placed
\end{itemize}

\subsubsection{TimeSystem}

% TODO
\begin{itemize}
    \item describe the update logic (mention the multiple comps updated)
    \item explain time event handling (and use)
    \item explain time event advancements
\end{itemize}

\subsubsection{TriggerSystem}

% TODO
\begin{itemize}
    \item describe the update logic
    \item explain how triggering works and why it works that way (factions)
    \item explain the general trap concept?
\end{itemize}

\section{Helpers}

% TODO
\begin{itemize}
    \item describe what helpers are and how are they used
    \item forward reference the not-singleton status of EntitySystem?
    \item explain why they were created (use in Lua and optionally in C++
        for better code readability)
    \item explain why they are slower than direct component manipulation
    \item explain component-helper relationship
    \item mention the general structure of a helper
    \item maybe talk about why they are namespaces and not classes?
        (+ namespace -> allows static class without any change basically)
\end{itemize}

\section{Script}

% TODO
\begin{itemize}
    \item talk about how it works (it's basically a wrapper facade)
    \item mention lpp::Exception
    \item explain the Lua C API (only informative? add a new section with a tutorial?
        tell the reader to read the Lua Programming Language 3rd edition book?)
    \item maybe go through some of the more complex functions?
\end{itemize}

\section{LuaInterface}

% TODO
\begin{itemize}
    \item talk about why it's needed (Lua needs static)
    \item talk about why it's centralised in one class?
\end{itemize}

\subsection{Initialising the API}

% TODO
\begin{itemize}
    \item explain the C++ function binding process
    \item explain the Lua module hierarchy
    \item explain other aspects besides function binding
\end{itemize}

\subsection{Extending the API}

% TODO
\begin{itemize}
    \item explain the general body of the interface functions (stack, return etc.)
    \item (super) small tutorial on how to add new interface functions
\end{itemize}

\section{GUI}

% TODO
\begin{itemize}
    \item talk about the GUI hierarchy
    \item explain initialization
    \item explain CEGUI button binding? maybe in general CEGUI manipulation?
    \item explain save/load file listing
    \item mention the GUIWindow class, base class of the following GUI windows
    \item explain why it has escape\_pressed handler and does not use CEGUI handlers
        for that (would be needed for all subwindows, too much a hassle - also would
        probably disregard the conditional closing)
\end{itemize}

\subsection{Console}

% TODO
\begin{itemize}
    \item talk about how awesome it is during debugging
    \item explain how execution and printing works (+ the history concept?)
    \item explain why it does not execute lines but multi lines after the execute
        button is pressed
\end{itemize}

\subsection{EntityCreator}

% TODO
\begin{itemize}
    \item mention that currently it is used for entity placement and
        the creation part is supposed to be a graphical way to make entities that
        is not currently in the game
    \item explain how it works
    \item explain how it's used for testing
    \item maybe mention that this is where the registered\_entity from EntitySystem
        is used
\end{itemize}

\subsection{EntityTracker}

% TODO
\begin{itemize}
    \item explain how tracking works
    \item talk about why it's useful
    \item mention the upgrade, exp convert and delete functionality?
\end{itemize}

\subsection{ResearchWindow}

% TODO
\begin{itemize}
    \item explain initialisation, how it works
    \item explain dummy\_unlock and its relationship with serialization
\end{itemize}

\subsection{Miscellaneous Windows}

% TODO
\begin{itemize}
    \item say that these are generally not that complex and quite straight forward
\end{itemize}

\subsubsection{BuilderWindow}

% TODO
\begin{itemize}
    \item talk about building registration (how it relates to unlocking)
    \item talk about the assembly line (also used in spell casting window)
        and why it's good (lack of images -> buttons need to be big and readable)
    \item don't forget to mention that this is just a graphical front end
        to EntityPlacer (with price management etc.) as is EntityCreator (though
        that one ignores price, and has all unlocked even enemies)
\end{itemize}

\subsubsection{GameLog}

% TODO
\begin{itemize}
    \item just mention that it is basically the ingame chat posting info
        for the player
    \item also maybe again mention history (same as console)?
\end{itemize}

\subsubsection{MessageToPlayerWindow}

% TODO
\begin{itemize}
    \item talk about button renaming and action assignment
    \item then mention how to use this as a whole
\end{itemize}

\subsubsection{OptionsWindow}

% TODO
\begin{itemize}
    \item talk about actions, key bindings and video settings
        (+ how they are done?)
\end{itemize}

\subsubsection{SpellCastingWindow}

% TODO
\begin{itemize}
    \item talk about spell registration (how it relates to unlocking)
    \item just mention the assembly line (already explained in BuilderWindow)
    \item talk about the relationship with the SpellCaster and how spell casting
        works on this side (simple invoking of the spell caster and marking
        the spell as active if needed)
    \item don't forget to mention that this is just a graphical front end
        to SpellCaster basically
\end{itemize}

\subsubsection{TopBar}

% TODO
\begin{itemize}
    \item just say something about its purpose (monitoring of resources)
\end{itemize}

\section{SpellCaster}

% TODO
\begin{itemize}
    \item discuss the spell casting concept in the game
    \item mention the spell types
    \item explain why it's so damn cool (ease of spell creation, run time spell creation)
    \item forward reference the Lua spell structure, which is explained later in Scripting
\end{itemize}

\section{Utilities}

% TODO
\begin{itemize}
    \item just say that these are classes that are not directly part of
        the game world but are used as background tools that support the game
\end{itemize}

\subsection{Camera}

% TODO
\begin{itemize}
    \item talk about how it allows free/nonfree mode, resetting
        and backups
    \item maybe talk more about the free nonfree mode and how to toggle them
        (both keybind and command)
\end{itemize}

\subsection{Effects}

% TODO
\begin{itemize}
    \item talk about how awesome they are and allowed the creation of
        extensible effect application framework in CombatSystem
    \item explain how to write one
    \item mention that they are used as template arguments and as such
        should conform the given interface (also explain that)
    \item maybe ilustrate their structure on one of them
\end{itemize}

\subsection{EntityPlacer}

% TODO
\begin{itemize}
    \item explain its purpose and how it works
    \item explain why only the graphics component data is used (so it's ignored
        by the game world and serializer, etc.)
\end{itemize}

\subsection{LevelGenerator}

% TODO
\begin{itemize}
    \item explain their purpose and how to write one
    \item mention the cycle count constructor parameter
    \item mention the default tables in config (and that it's easy to add new ones)
\end{itemize}

\subsubsection{RandomLevelGenerator}

% TODO
\begin{itemize}
    \item say that this is just naive algorithm for a pseudo random level generation
        using number of neighbours that are gold deposits to determine if a gold
        deposit should be spawned
\end{itemize}

\subsection{RayCaster}

% TODO
\begin{itemize}
    \item explain that I'm a noob that stole other programmer's idea for this
        from the Ogre wiki
    \item explain why it's needed (polygon precision ray casting -> half cubes,
        without it only bounding boxes are checked and free space is impenetrable)
\end{itemize}

\subsection{SelectionBox}

% TODO
\begin{itemize}
    \item explain its purpose and how it works
    \item talk about single/area selection and multi selection using shift
\end{itemize}

\section{Pathfinding}

% TODO
\begin{itemize}
    \item recapitulate form analysis that a modifiable function was used
        that allows different algorithms, heuristics and path types
    \item mentions its parameters (destruction, add path are primary!)
    \item mention how destruction works
    \item mentions why path addition is optional (for checks)
\end{itemize}

\subsection{Algorithms}

% TODO
\begin{itemize}
    \item explain how they should be implemented (what functions, as they
        are used as template parameters)
    \item mention that A* is the only currently implemented
    \item mention portal implementation
    \item mention difference in complexity (component lookups)
    \item mention any other differences from a general A*
\end{itemize}

\subsection{Heuristics}

% TODO
\begin{itemize}
    \item explain how they are used and why they are inheriting a base class
        while other functors are not (state needed for some -> run away heuristic)
    \item maybe demonstrate on an example as the code is small?
\end{itemize}

\subsection{Path Types}

% TODO
\begin{itemize}
    \item explain what these things are and when they are used to check
        if the algorithm should stop
    \item explain their effect on the A* algorithm
\end{itemize}

\section{Serialization}

% TODO
\begin{itemize}
    \item talk about how easy component serialization is and how
        to create template specializations for new components
    \item show an example?
    \item explain the whole saving process, ents to be destroyed, unlocks,
        player, grid etc.
    \item explain the loading process
\end{itemize}

\section{Player}

% TODO
\begin{itemize}
    \item just say that it is used as a resource bank, keeping track
        of gold, mana, units etc
    \item mention that it also holds the starting unlocks used for new games
        (saved there during the game initialisation)
\end{itemize}

\section{Singleton Design Pattern}

% TODO
\begin{itemize}
    \item explain what this stuff is, its pros \& cons
    \item mention that the main reason for its use in this game
        is having one instance, not that much global access
\end{itemize}

\subsubsection{Script}

% TODO
\begin{itemize}
    \item mention how two Lua virtual machines cannot communicate
        and data are bound to C++ (via EntitySystem) so it's completely
        unnecessary to have more than one Scripting engine
\end{itemize}

\subsubsection{GUI}

% TODO
\begin{itemize}
    \item why would we want to have two GUIs?
    \item mention that here is the global access very good for testing?
\end{itemize}

\subsubsection{Player}

% TODO
\begin{itemize}
    \item why would we want two sets of player resources?
    \item serialization preserves them and shown is only the main set
        (also used is only the main one)
\end{itemize}

\subsubsection{Grid}

% TODO
\begin{itemize}
    \item explains how it only operates on a set of given IDs that are
        made during graph creation (+ mentions it would be easy to implement
        their switch) so there does not need to be more than one pathfinding grid
    \item also mention that due to the nature of the game levels, it would be nonsense
        to have two grids
\end{itemize}

\subsubsection{EntitySystem}

% TODO
\begin{itemize}
    \item explain how even though it's getting passed around a lot (mainly Helpers),
        there could be a reason to use more than one EntitySystem instance (like a backup
        for example) and thus the Singleton pattern might not be the best choice
    \item maybe compare it to the 4 previous classes
\end{itemize}

\section{Extending the Game}

% TODO
\begin{itemize}
    \item small sequence of tasks that are needed for a new feature addition
        (that is component, system etc.)
    \item idea: show how a TargetedComponent would be implemented, allowing
        handlers for targeting/untargeting and how this could be used to create
        a chess mod
\end{itemize}

\chapter{Scripter's Documentation}

While we cannot change parts of the engine without recompilation of its source code, we can provide implementation of various parts of the
game through scripts written in the Lua programming language\footnote{To learn about the language, we recommend the Programming in Lua
book~\cite{PIL}.}. These scripts are mainly used during initialization and for the definition of
entities, enemy waves, research nodes and spells. In this section we will examine how to write these parts of the game.

A reference documentation for our Lua API can be found in the \texttt{lua-api} directory which is a part of the Attachment~B and all
of the game's scripts can be found in the \texttt{src/scripts} directory which is a part of the Attachment~A.

\section{Initialization}

When the game starts, it executes two script files -- \texttt{config.lua} and \texttt{init.lua}. The \texttt{config.lua} contains
a Lua table\footnote{A table is the only data structure in Lua, it is an associative array of dynamic size that can be indexed
by any value except for \texttt{nil}, which represents no value.}
\texttt{config} with values used by the engine such as time periods and multipliers, blocks that are used by the default
level generator and a list of directories that contain scripts that are to be loaded when the game starts. The purpose of each value
is documented within this script file.

The file \texttt{init.lua} is used mainly to load other script files and user created mods. Additionally, it loads the values stored
in the configuration file to the engine. We can also use this file for any auxiliary commands we would like to be executed at the
start of the game -- e.g. for testing purposes.

If we want to modify our game we have two options. Our first option is to edit the already implemented Lua scripts that are located in the
\texttt{scripts} directory. This also includes creating new files and adding their names to the list of scripts that are to be loaded,
which can be found in the file \texttt{scripts/core.lua}. Our second option is to create a new scripting module, which is done by creating
a new directory in the game's root directory\footnote{The directory where the game's executable file is located.}
and adding this directory to our configuration script. This new directory has to contain a
script called \texttt{core.lua}, which will be executed by the game at startup. Note that scripting modules are executed precisely in
the order in which they are listed in the configuration script.

In Listing~\ref{core-lua-ex}, we can see an example of a small \texttt{core.lua} script located in the directory \texttt{some-mod}
in the game's root directory. It loads two additional scripts using the function \texttt{game.load}, which takes the path
to a scripts file which is relative to the game's root directory. Additionally, it redefines the function \texttt{game.init\_level},
which is called whenever the game creates a new empty level but before level generation starts. The important part of this function
is its return value -- if it is not \texttt{true}, the level generation will not happen. We can use this to write our own level
generator in Lua, which we would call from inside this function and then we would prevent the level generator written in \cpp by
returning \texttt{false}.

\begin{listing}[H]
    \centering
    \begin{lstlisting}
-- Scripts that will be loaded.
local scripts = {"script-1.lua", "script-2.lua"}

-- Path to this module from game root directory.
local path = "some-mod/"

-- Load the scripts, .. is used for string
-- concatenation.
for idx, script in ipairs(scripts) do
    game.load(path .. script)
end

-- Redefine the level init function.
game.init_level = function(width, height)
    game.print("A new level of some mod created!")
    game.print("Dimensions: " .. width .. ", " .. height)
    return true
end

-- We can execute any Lua code.
game.print("Some mod loaded!")
    \end{lstlisting}
    \caption{An example \texttt{core.lua} script.}
    \label{core-lua-ex}
\end{listing}

Scripts loaded in a \texttt{core.lua} script can execute any valid sequence of Lua commands and calls to our modding API, but they
are mainly used for the definition of new entities, enemy waves, research nodes and spells, which we will now examine.

\section{Entities}

To create a new entity, we need to create a Lua table that defines its components. In Listing~\ref{ent-table-ex1} we can see an example
of a simple ogre entity. Each entity definition needs to contain a table called \texttt{components}, which is a list of all component
types this entity has and is used by the game whenever it creates a new instance of this entity. For easier listing of components, the
file \texttt{scripts/enum.lua} -- which is loaded by default by the \texttt{scripts/core.lua} scripts -- contains an enum table called
\texttt{component} which contains variables representing the type identifiers of each component.

Once we list all of the components our new entity has, we need to define all of its components that require us to. In the file
\texttt{components.txt}, which is a part of the Attachment~B, we can find which components need to be defines in the entity definition
table and what fields do these component tables need to have. Note that the order in which we define component does not matter.

For this particular example entity, we first define the \texttt{PhysicsComponent}, which represents the physical presence of an entity
inside of the game world. It has a single field called \texttt{solid}, which should be \texttt{true} for buildings that cannot be
walked through and \texttt{false} for any other entity. Then we define the \texttt{HealthComponent}, which represents the health of an
entity and its ability to be damaged and killed. It has three fields that need to be set -- \texttt{max\_hp}, which determines the
maximum amount of health the entity can have, \texttt{regen} which determines how much health should the entity gain on each regeneration
period of the health system, and \texttt{defense}, which determines the amount of damage that should be subtracted from incoming damage.

Once we define all of the components of our entity, we can register her in the list of all defined entities using the function
\texttt{game.entity.register}.

\begin{listing}
    \centering
    \begin{lstlisting}
-- Table defining the ogre entity.
ogre = {
    -- List of components the ogre entity has.
    components = {
        game.enum.component.physics,
        game.enum.component.health
    },

    -- Definition of the physics component.
    PhysicsComponent = {
        solid = false
    },

    -- Definition of the health component.
    HealthComponent = {
        max_hp = 1000,
        regen = 10,
        defense = 50
    }
}

game.entity.register("ogre")
    \end{lstlisting}
    \caption{Definition of a simple entity.}
    \label{ent-table-ex1}
\end{listing}

\subsection{Blueprints}

Components can also define the behavior of an entity. To implement this, the game contains the concept of \emph{blueprints}. Blueprint
is a table that contains functions that describe the behavior of an entity of a specific type. Which blueprint is used is defined
for each component individually, this allows us to combine the behavior of different entity types into a new one without the need
to create new blueprints.

In Listing~\ref{ent-table-ex2} we can see an implementation of an example healer entity. This entity has three components that describe
its behavior. \texttt{SpellComponent} describes the ability to cast spells and contains fields \texttt{blueprint}, which determines
which blueprint is used for spell casting, and \texttt{cooldown}, which determines the minimal time between two consecutive spell casts.
\texttt{AIComponent} describes the overall behavior the entity performs on every periodic AI update and contains the field
\texttt{blueprint}, which determines which blueprint is used for these AI updates. \texttt{OnHitComponent} describes how the entity
reacts when it gets attacked by an enemy and contains fields \texttt{blueprint}, which determines which blueprint is used to find the
reaction to enemy attacks, and \texttt{cooldown}, which determines the minimal time between two consecutive reactions -- this can be used
to prevent overflow of the game's log if the entity just notifies the player of the enemy attack.

\begin{listing}[H]
    \centering
    \begin{lstlisting}
cowardly_healer = {
    components = {
        game.enum.component.spell,
        game.enum.component.ai,
        game.enum.component.on_hit
    },

    SpellComponent = {
        blueprint = "healer_blueprint",
        cooldown = 25.0
    },

    AIComponent = {
        blueprint = "healer_blueprint"
    },

    OnHitComponent = {
        blueprint = "coward_blueprint",
        cooldown = 1.0
    }
}

game.entity.register("cowardly_healer")
    \end{lstlisting}
    \caption{An example of component definitions that use blueprints.}
    \label{ent-table-ex2}
\end{listing}

The entity behaves in the same way as a healer does when it casts a spell or has its AI updated, so these two components use the
\texttt{healer\_blueprint}, the implementation of which can be seen in Listing~\ref{ent-table-ex3}. This blueprint contains the
implementation of function used by all of the components our \texttt{cowardly\_healer} has, but the \texttt{on\_hit} function
is not used in this case because a regular healer entity heals itself and continues its normal behavior when it is attacked while this entity
is too cowardly to continue and simply runs away.

\begin{listing}[H]
    \centering
    \begin{lstlisting}
-- Functions describing the behavior of a healer.
healer_blueprint = {
    -- Associated with the spell component.
    cast = function(id)
        -- Heal all friends of entity <id> that are nearby.
    end,

    -- Associated with the ai component.
    update = function(id)
        -- If possible, heal friends. Otherwise, attack enemies.
    end,

    -- Associated with the on hit component.
    on_hit = function(id, enemy)
        -- Heal self.
    end
}
    \end{lstlisting}
    \caption{Implementation of the healer blueprint.}
    \label{ent-table-ex3}
\end{listing}

Because of this, our \texttt{cowardly\_healer} uses the \texttt{coward\_blueprint}, the implementation of which can be seen in
Listing~\ref{ent-table-ex4}, for its \texttt{OnHitComponent}. The \texttt{on\_hit} function of this blueprint then simply forces our
\texttt{cowardly\_healer} to run away from its attacker whenever it gets attacked.

Since the individual functions may differ in both name and parameters for the different components that use blueprints, this information
is included in the component description in the file \texttt{components.txt}, which is a part of the Attachment~B.

Note that the implementation of these behavior blueprints is not required. However, because of the way the game operates to support
blueprints, every of these behavioral function needs to be contained within a Lua table whose name is set in the component's
\texttt{blueprint} name and has to have the signature that is required when it is a part of a blueprint.

\begin{listing}[H]
    \centering
    \begin{lstlisting}
-- Functions describing the behavior of a coward.
coward_blueprint = {
    -- Associated with the on hit component.
    on_hit = function(id, enemy)
        -- Run away from entity <enemy>.
    end
}
    \end{lstlisting}
    \caption{Implementation of the coward blueprint.}
    \label{ent-table-ex4}
\end{listing}

\section{Enemy waves}

The \emph{wave table} is a Lua table that defines the composition of enemy waves and delays between them. Each wave table contains the
\texttt{init} function, which is called whenever a level that uses the table is created. Additionally, it contains a pair of functions
\texttt{wstart\_X} and \texttt{wend\_X} for each of its waves, where \texttt{X} starts at 0 and gets incremented for each wave.

An example of the \texttt{init} function of a wave table can be seen in Listing~\ref{wave-table-init}. In it, the wave table resets the wave
composition that was set during any previous wave sequence by calling the function \texttt{game.wave.clear\_entity\_blueprints}
\footnote{In this case, the term blueprint does not refer to the blueprints we have discussed in the previous section, but to an
entire entity definition.}. It then sets the number of waves with \texttt{game.wave.set\_wave\_count}, resets the current wave number with
\texttt{game.wave.set\_curr\_wave\_number} and changes the time before the first wave starts with \texttt{game.wave.set\_countdown}.

\begin{listing}[H]
    \centering
    \begin{lstlisting}
wave = {
    -- Initializes this wave table.
    init = function()
        game.wave.clear_entity_blueprints()
        game.wave.set_wave_count(2)
        game.wave.set_curr_wave_number(0)
        game.wave.set_countdown(300)
    end
}
    \end{lstlisting}
    \caption{An example of the intialization function in a wave table.}
    \label{wave-table-init}
\end{listing}

In Listing~\ref{wave-table-first}, we can see an implementation of the starting and ending function of a first wave within a wave table.
When the first wave starts, the function \texttt{wstart\_0} is called. The starting functions generally define the composition
of the wave that is starting using the function \texttt{game.wave.set\_entity\_total}, which tells the wave system how many entities
comprise this wave so that it knows when to end the wave, and the function \texttt{game.wave.add\_entity\_blueprint}, which creates a new
member of the wave.

The entities that are part of the wave spawn on specific nodes that were set during level generation. If there are more entities in a wave
than there are spawning nodes, the wave system only spawns enough entities to cover these nodes at a time. Before it spawns another
group of entities, it waits a specific time period, which can be set using \texttt{game.wave.set\_spawn\_cooldown}.

Once all entities that belong to the first wave are killed, the \texttt{wend\_0} function is called, which generally only changes the
time before the next wave. If this time is not changed, the previously set value is used.

\begin{listing}[H]
    \centering
    \begin{lstlisting}
wave = {
    -- Called when the first wave starts.
    wstart_0 = function()
        game.wave.set_entity_total(2)
        game.wave.add_entity_blueprint("ogre")
        game.wave.add_entity_blueprint("coward_healer")
        game.wave.set_spawn_cooldown(10.0)
    end,

    -- Called when the first wave ends.
    wend_0 = function()
        game.wave.set_countdown(180)
    end
}
    \end{lstlisting}
    \caption{An example of the first wave definition in a wave table.}
    \label{wave-table-first}
\end{listing}

Following waves are defined similarly, but note that if we do not call the function \texttt{game.wave.clear\_entity\_blueprints} between
waves, the entity blueprints of the previous wave are not deleted and will also be included in the next wave. This can be used to create
growing groups with each wave without the need to re-add the entities.

\begin{listing}[H]
    \centering
    \begin{lstlisting}
wave = {
    -- Called when the second wave starts.
    wstart_1 = function()
        game.wave.set_entity_total(4)
        game.wave.add_entity_blueprint("ogre")
        game.wave.add_entity_blueprint("coward_healer")
        game.wave.set_spawn_cooldown(10.0)
    end,

    -- Called when the second wave ends.
    wend_1 = function()
        game.print("All enemies defeated!")
        game.print("Now here's even more enemies!")
        game.wave.turn_endless_mode_on()
    end
}
    \end{lstlisting}
    \caption{An example of the second wave definition in a wave table.}
    \label{wave-table-second}
\end{listing}

In Listing~\ref{wave-table-second} we can se the definition of the second and, in this case, the last wave defined within a wave table.
Its \texttt{wend} function does not call the blueprint resetting function so the total number of entities is set to four, because the two
entities from the previous wave also spawn in this wave.
If we want the wave sequence to repeat its last indefinitely wave once all waves are finished, we can use the function
\texttt{game.wave.turn\_endless\_mode\_on} in the last \texttt{wend} function.

\section{Research}

The research nodes in the game are organized to a grid of six rows and seven columns. The game controls these nodes by passing the number
of the row and of the column the node is located in into three functions. The first of these functions, \texttt{game.gui.research.get\_name}
is called once at the start of the game and returns the name of the node located at the position passed as its parameters. The second
function, \texttt{game.gui.research.get\_price}, returns the price of the unlock of the research node and the third of these functions,
\texttt{game.gui.research.unlock} is called when the player buys the research node and performs the unlock of the node's benefit to the
player.

The unlocking function does not impose any limitations on the characteristic of the node. It can unlock new spell, new minion, new building
or perform a one time action that gives the player a bonus of sorts. An example of the research implementation can be found in the
file \texttt{scripts/research.lua}, but the game does not have any requirements on the implementation besides the functionality of these
three functions.

Note that the \texttt{game.gui.research} table, which should contain these functions, is already predefined by the game and as such we should
not overwrite it when we write our research node.

\section{Spells}

Similarly to the definition of a research node, spells are defined by a table that contains three functions and is itself contained within
the table \texttt{game.spell.spells}. Note that, unlike the research table, this table is not created by the engine but is created in the
script \texttt{scripts/spells.lua}. This means, that if we create a modification of the game, we can simply add new spell definitions
to this predefined table, but if we create our own replacement of the \texttt{scripts/spells.lua}, we need to create this table ourselves.

When the player selects a spell, the engine calls the function \texttt{init}, located in the spell table. This function should change the
player into casting state, often done by setting the type of the currently cast spell. When the player casts a spell that has been
previously initialized, the engine calls the two remaining function that are in the spell table.

Firstly, it calls \texttt{pay\_mana},
which is supposed to subtract the mana cost from the player and to return \texttt{true} if the player can cast the spell. If this function
returns \texttt{false}, the spell casting is interrupted and the player is notified that he has insufficient mana.

Secondly, the engine calls the function \texttt{cast} if the spell casting was not interrupted during the call to \texttt{pay\_mana}. This
function performs the actual spell cast. When we create a new spell, we can allow the player to use it by calling the function
\texttt{game.spell.register\_spell}, which accepts the name of the spell table -- without the \texttt{game.spell.spells} prefix -- as its
parameter, either directly inside a script file or inside a research node. Note that, unlike the previous two functions which have no
parameters, this function has different parameters for different spell types.

\subsubsection{Targeted spells}

In Listing~\ref{spell-ex-targeted}, we can see an example of one of these spell types -- a targeted spell. This type of spell  affects a
single currently selected entity. In its \texttt{init} function, it changes the type of the current spell to \texttt{targeted} using
the function \texttt{game.spell.set\_type}. The \texttt{targeted} type, along with other spell types, is a variable stored in the
table called \texttt{game.enum.spell\_type}. These enum tables are used in a similar way to \cpp enums that are defined in the source file
\texttt{Enums.hpp} -- the fields of these tables hold integer values that correspond with the \cpp values.

The \texttt{pay\_mana} function subtracts mana from the player using the function \texttt{game.player.sub\_mana}. This API function returns
\texttt{true} if the player has enough mana and false otherwise so the returned value can be returned from \texttt{pay\_mana} as well.
Because the implementation of this function is almost always the same, we will not list it in any of the other spell examples, but it is
always required regardless of spell type.

The \texttt{cast} function of a targeted spells accept the identifier of the selected entity as its parameter. In this case, it teleports
the target to a random unobstructed place in the game world and then uses the function \texttt{game.spell.stop\_casting}. This causes the
spell to be interrupted once it has be cast once, if we want to allow consecutive casts of the same spell, we can simply avoid using this
API function and let the player to decide when to stop casting.

\begin{listing}[H]
    \centering
    \begin{lstlisting}
-- Teleports the target to a random location.
game.spell.spells.random_teleport = {
    -- Initialization, sets the spell type.
    init = function()
        game.spell.set_type(game.enum.spell_type.targeted)
    end,

    -- Subtracts mana from the player and returns true
    -- if the player had enough, returns false otherwise.
    pay_mana = function()
        return game.player.sub_mana(50)
    end,

    -- Applies the effect of the spell to the targeted
    -- entity, which has identifier <target>.
    cast = function(target)
        game.grid.place_at_random_free_node(target)
        game.spell.stop_casting()
    end
}
    \end{lstlisting}
    \caption{An example of a targeted spell.}
    \label{spell-ex-targeted}
\end{listing}

\subsubsection{Positional spells}

In Listing~\ref{spell-ex-positional}, we can see an example of a positional spell. This type of spells has an effect at a specific point in
the game world. The \texttt{cast} function accepts two dimensional coordinates as its parameters, which specify this point with no regard
to height. It uses the \texttt{game.command.reposition} function, which commands the minion that is closest to the specified position to
move to the position, and then interrupts the cast to avoid the accidental repositioning of multiple minions.

This type of spells can be used for commands similar to this one, to spawn groups of entities or a single entity. However, since the
\texttt{init} function only sets the type of the spell and nothing else, the player would not see the single entity that is supposed
to be spawned following the mouse cursor as is common in many games. For this, the spell type presented in the following section is
more suitable.

Note that since the spell cannot be seen during cast and is activated when the player clicks somewhere within the game world, it's advised
to interrupt casting inside the \texttt{cast} function to avoid accidental repeated casts.

\begin{listing}[H]
    \centering
    \begin{lstlisting}
-- Orders the minion that is closest to the area of
-- the spell cast to move to that area.
game.spell.spells.order_repositioning = {
    -- Initialization, sets the spell type.
    init = function()
        game.spell.set_type(game.enum.spell_type.positional)
    end,

    -- Applies the effect of the spell, accepts
    -- two dimensional coordinates in the game world
    -- which denote the place where the player cast
    -- the spell.
    cast = function(x, y)
        game.command.reposition(x, y)
        game.command.stop_casting()
    end
}
    \end{lstlisting}
    \caption{An example of a positional spell.}
    \label{spell-ex-positional}
\end{listing}

\subsubsection{Placing spells}

In Listing~\ref{spell-ex-placing}, we can see an example of a placing spell. This type of spells is used when we want the spell to place a
new entity into the game world. In the \texttt{init} function, we -- besides setting the spell type -- invoke the entity placer by using
\texttt{game.entity.place} which accepts the name of an entity defining table as its argument and creates a model of the entity that follows
the mouse cursor. From this point onward, the placement is
performed by the engine and the \texttt{cast} function is only called after the entity has been placed and is provided with the identifier
of the placed entity as its argument so it can manipulate with it.

In this case, we spawn our \texttt{coward\_healer} entity and lower its health. Note that since we did not call
\texttt{game.spell.stop\_casting}, the placement process will continue until canceled or the player runs out of mana, but unlike positional
spells, placing spells provide visual hint of the ongoing cast -- the model of the entity that follows the mouse cursor.

\begin{listing}
    \centering
    \begin{lstlisting}
game.spell.spells.spawn_damaged_coward_healer = {
    -- Initialization, sets the spell type.
    init = function()
        game.spell.set_type(game.enum.spell_type.placing)
        game.entity.place("coward_healer")
    end,

    -- Does NOT apply the effect of the spell,
    -- that has been done by the engine.
    -- This function only manipulates the placed
    -- entity.
    cast = function(id)
        game.health.sub(id, 100)
    end
}
    \end{lstlisting}
    \caption{An example of a placing spell.}
    \label{spell-ex-placing}
\end{listing}

\subsubsection{Global spells}

The last type of spells that we can define are global spells, the example of which can be seen in Listing~\ref{spell-ex-global}. These
spells serve as Lua functions that are executed when the player casts them. This particular spell increases the amount of gold that the
player has whenever it is cast using the \texttt{game.player.add\_gold} API function.

These spells are cast on every release of the left mouse button, so the player may accidentally cast these spells multiple times because
they offer no visual hint of the ongoing cast. Because of this, using \texttt{game.spell.stop\_casting} to interrupt the casting process
in the \texttt{cast} function is advised.

\begin{listing}
    \centering
    \begin{lstlisting}
    game.spell.spells.cheat = {
        -- Initialization, sets the spell type.
        init = function()
            game.spell.set_type(game.enum.spell_type.global)
        end,
        
        -- Applies the effect of the spell.
        cast = function()
            game.player.add_gold(10000)
            game.spell.stop_casting()
        end,
    }
    \end{lstlisting}
    \caption{An example of a global spell.}
    \label{spell-ex-global}
\end{listing}

\chapter{User's Documentation}

This section presents our game to the player and explains its installation, startup and controls. The playable version of our game
can be found on the attached DVD as Attachment~C.

\section{Installation and startup}

The game requires Windows~7 or a newer Windows operating system to run. Additionally, it requires a graphics card that is compatible
with either OpenGL~3 or Direct3D 9.

To install the game, we should first move the directory that contains the game's executable file anywhere on our hard disk. Although this is
an optional step, it allows our game to create new files which allows us to use the save feature of the game. Next, we need to install
the Visual C++ Redistributable, which can be done by executing the file \texttt{deps/vc\_redist.x86.exe}.

Once the redistributable is installed, we can start the game by executing the file \texttt{tdt.exe}. If this is the first startup, we will
see a window that allows us to configure our graphics options, which can be seen in Figure~\ref{ogre-config}. In this window, we need to
choose either the OpenGL or Direct3D from a drop-down list labeled \emph{Rendering Subsystem}. Once we choose our rendering subsystem, we
will be offered with various graphical settings we can change in the bottom list labaled \emph{Rendering System Options}. We will now also
be able to start the game by clicking on the \emph{OK} button.

\begin{figure}[H]
    \centering
    \includegraphics[width=10cm]{../img/ogre-cfg.png}
    \caption{Initial graphics setting window that is shown on the first start of the game.}
    \label{ogre-config}
\end{figure}

Next time we start the game, this window will not appear and the game will use the settings we chose the first time. If we want to change
these settings, we can force this window to appear on the next start of the game by deleting the file \texttt{ogre.conf}.

\section{Main menu}

\begin{figure}[H]
    \centering
    \includegraphics[width=10cm]{../img/gui-intro.png}
    \caption{Initial menu that we can see when we start the game.}
    \label{gui-intro}
\end{figure}

When the game starts, we will be greeted by the game's main menu, which can be seen in Figure~\ref{gui-intro}. In this menu, we can create
a new game by clicking on the button \emph{NEW GAME}, which shows a window that will prompt us for level dimensions. Here, we can either
input any pair of numbers in the given range, or choose one of the buttons with predefined dimensions.

If we already have a previously saved game, we can click the button \emph{LOAD GAME}, which will show a window all saved games that are
located in the \texttt{saves} directory. Aside from starting a level -- be it a new one or a loaded one -- we can view the options menu by
clicking on the button labeled \emph{OPTIONS} or close the game with the \emph{QUIT} button.

\section{User interface}

When we get into the game, we have a similar view to the one shown in Figure~\ref{gui-full}. At the top of the screen, we can see the top
bar, which displays our current resources -- our gold, our current and maximum mana with mana regeneration in parentheses, current number
of minions we control and the total number of minions we have, which includes minions that died and are respawning. Next to these resources,
we can also see the current system time.

Below the top bar, we can see the game world. The game world consists of blocks, which represent walls and buildings, our minions and
attacking enemies. In the image we can see a starting area of a newly generated world, which includes a free area in the middle where
the player's dungeon throne -- which represents the life of the player -- is placed along with some other starting building. We can select
any entity by either clicking on it with our left mouse button or by pressing the mouse button and dragging the mouse which allows us
to select multiple entities.

At the bottom of the screen, we can see three large windows. The one of the left is the tool bar. This window can contain three different
tool bars that can be switched between by clicking on one of the buttons located in the top row of the tool bar. The menu shown by default
is used to save and load the game and to show the options, main menu and research window. If we click on the \emph{SPELL} button the tool bar
changes to the spell bar, which can be used to cast spells, and if we click on the \emph{BUILD} button the tool bar changes to the building
bar, which can be used to place new buildings. The last button on the top row, \emph{MENU}, returns us to the initial menu bar.

Next to the tool bar is the game log, which shows messages from the game and from the player's minions. These message can for example include
warnings about an attack or notifications about insufficient resources.

The rightmost of these three windows is the entity viewer, which shows information about the currently selected single entity -- note that
it will not show anything when we select multiple entities. This information includes data such as the health, mana, level and name of the
selected entity. Besides the information shown, the entity viewer contains two buttons. The left button can be used to convert gold to
experience, which can be used to quickly level the entity up, and the right button can be used to sacrifice our entity in return for a part
of its cost.

Above the entity viewer window, we can see a small countdown bar, which tells us the time before the next wave of enemies will attack our
dungeon.

\begin{figure}[H]
    \centering
    \includegraphics[width=\textwidth]{../img/gui-full-text.png}
    \caption{The game screen.}
    \label{gui-full}
\end{figure}

\section{Goal of the game}

In our game, the main goal of the player is to protect its dungeon -- specifically the dungeon throne -- from waves of enemy attackers
\footnote{Note that this applies to the game itself, mods are free to change the winning or losing conditions.}.
If the dungeon throne is destroyed, the player loses. If, on the other hand, the throne survives all of the enemy waves, the player wins.

To protect the dungeon throne, the player can place buildings in exchange for gold gathered by miners and cast spells that can damage or slow
the enemies in exchange for mana, which automatically regenerates. When a new level is created, the player starts with a small area that
contains the dungeon throne, a mine that spawns a miner and a gold vault that the miner uses for gold storage. Additionally, one side of a
map will contain enemy spawners, which will spawn enemies in intervals defined by the game's wave system.

Some of the buildings can spawn minions that will defend the dungeon from enemies and will revive these minions when and if they are killed.
To unlock these and other buildings, as well as new spells, the player can use the research window that is accessible from the menu tool
bar to exchange gold for new research unlocks that can provide buildings, spells or one time bonuses to the player.

Once all enemy waves are beaten, the player will be presented with the option to play in a sandbox mode, which allows them to continue
building their dungeon without any further enemy waves, or in an endless mode, which will repeat the last enemy wave indefinitely.

\section{Research}

Research is the tool we can use to unlock new buildings, spells and bonuses. The research window, which can be seen in
Figure~\ref{gui-research}, contains six rows that have seven tiers of unlock nodes each. To unlock a node, one can click on the research
button that represents it and, if they have enough gold, the unlock takes effect.

There are three types of unlocks in the game -- one time bonuses, building unlocks and spell unlocks. To avoid redundancy, we will
list one time bonuses with their effects here and the rest of the unlocks -- that is, buildings and spells -- will be described in the
following chapters.

\begin{itemize}
    \item \emph{KILL ALL ENEMIES} -- kills all enemy entities in the world.
    \item \emph{INCREASE PROD.} -- increases the production limit of all friendly buildings by one.
    \item \emph{DOUBLE PROD.} -- doubles the production limit of all friendly buildings by one.
    \item \emph{LEVEL UP} -- increases the level of all friendly entities.
    \item \emph{UBER THRONE} -- heals the dungeon throne and increases its health and defense.
    \item \emph{INSTANT PROD.} -- causes all buildings to spawn minions without waiting.
\end{itemize}

The research nodes have unified unlock cost per tier. The cost is 200, 400, 600, 800, 1000, 1500 and 2000 gold for tiers 1, 2, 3, 4, 5, 6 and
7 respectively. When a research node is unlocked, its name gets prefixed with a plus symbol and the next research node in the row is revealed.

\begin{figure}[H]
    \centering
    \includegraphics[width=\textwidth]{../img/gui-research.png}
    \caption{Research window which contains nodes that can be unlocked.}
    \label{gui-research}
\end{figure}

\section{Spell casting}

Spells are the means for the player to directly affect the battles between their minions and attacking enemies. Most spells are unlocked
through the research window\footnote{Those that are not are available to the player from the start of the game.}, in which a research
node \emph{SPELL NAME} unlocks spell \emph{spell\_name}.

We can cast spells through the spell tool bar window, which can be shown by clicking on the \emph{SPELL} button that is located in the
top row of buttons in the tool bar window. We can see a picture of the spell tool bar in Figure~\ref{gui-spell}. From top to bottom, it
contains a row of buttons that can be used to switch to other tool bars, a row of four buttons that represent different spells and two
buttons on the bottom row that are used to switch between spells.

\begin{figure}[H]
    \centering
    \includegraphics[width=\textwidth]{../img/tool-spell.png}
    \caption{Window that allows the player to cast spells.}
    \label{gui-spell}
\end{figure}

We can think of the spell tool bar as of a conveyor belt which contains our spells in the order we have unlocked them. The
\emph{\textless\textless\textless} button moves the belt to the left by one spell and the \emph{\textgreater\textgreater\textgreater}
button moves it to the right by one spell. Once we locate the spell we would
like to cast, we can either click on it with the left mouse button or press the key assigned to it -- these four spell buttons correspond
to key binding actions \emph{SPELL/BUILD 1-4} from left to right. Once we select a spell for casting, a label reading \emph{ACTIVE} will
appear under the name of the currently casted spell as can be seen below \emph{spawn\_imp} in the figure above. This is because some of
the spells that are in the game do not have any visual effects that would indicate the casting process.

\subsection{Spells}

There are four different type of spells that differ in the way they are cast. In the following sections we will examine each of these
spell types and list all of their members. In the spell lists, we can see the name of the spell followed by its mana cost in parentheses
and its description.

\subsubsection{Targeted spells}

To cast a targeted spell, we first need to select an entity that will serve as the target for the spell's effect, which we can do by left
clicking on the entity in the game world. Once we have a target, we can cast the spell wich will immediately apply its effect to the
selected target. Targeted spells are:

\begin{itemize}
    \item \emph{attack} (0) -- commands the closest combat minion with the smallest amount of assigned tasks to attach the selected
        entity.\footnote{This spell is actually global in implementation, but its effect is very similar to that of targeted spells
        so it is placed in this list.}
    \item \emph{heal} (10) -- heals a single currently selected friendly entity to full health.
    \item \emph{slow} (20) -- halves the speed of the enemy target for five seconds.
    \item \emph{lightning} (20) -- strikes the enemy target with a bolt of lightning.
    \item \emph{freeze} (40) -- freezes the enemy target in place for five sconds.
    \item \emph{chain\_lightning} (60) -- strikes the enemy target with a bolt of lightning that bounces to nearby enemies.
    \item \emph{teleport} (100) -- teleports the enemy target to a random place in the game world.
    \item \emph{destroy\_block} (200) -- destroys a neutral mineable target, e.g. a wall.
\end{itemize}

\subsubsection{Global spells}

Global spells generally have an instant global effect, though some of them can function similarly to targeted spells but work with
multiple selected entities. Global spells are:

\begin{itemize}
    \item \emph{mine} (0) -- commands the closest miner with the smallest amount of assigned tasks to mine any selected
        mineable entities.
    \item \emph{return\_gold} (0) -- commands all minions that carry gold to return it to the closes gold vault.
    \item \emph{fall\_back} (0) -- commands all minions to return to their spawners.
    \item \emph{meteor\_shower} (200) -- spawns five meteors at random places in the game world that impact the ground and cause
        an explosion.
    \item \emph{lightning\_storm} (500) -- strikes up to thirty enemies with a lightning bolt that bounces to nearby enemies.
\end{itemize}

\subsubsection{Placing spells}

Placing spells are used to place a single entity into the game world. Once we start casting a placing spell, the placed entity will start
to follow the mouse cursor giving us visual hint of its placement. Placing spells are:

\begin{itemize}
    \item \emph{spawn\_imp} (20) -- places an imp that will defend the dungeon from enemies for two minutes.
    \item \emph{meteor} (40) -- places a visual marker on the ground that will then be hit with a meteor which causes an explosion.
    \item \emph{healing\_wave} (30) -- places an expanding orb of light that heals all minions in its area to full health.
    \item \emph{slowing\_wave} (50) -- places an expanding orb that halves the speed of all enemies in its area for five seconds.
    \item \emph{freezing\_wave} (70) -- places an expanding orb of ice that freezes all enemies in its are in place for five seconds.
    \item \emph{portal} (200) -- places two portals on the ground that allow fast transportation between them. Note the portal must be
        placed twice within one spell cast..
\end{itemize}

\subsubsection{Positional spells}

Positional spells are used similarly to placing spells, but lack the visual hint of an entity following the mouse cursor. Once we select
the spell to cast, we can click in the game world with the left mouse button to apply the effect of the spell. Positional spells are:

\begin{itemize}
    \item \emph{reposition} (0) -- commands the closest entity with fewest tasks assigned to it to move to the selected position.
    \item \emph{spawn\_imp\_gang} (100) -- spawns a gang of four imp gang members and one imp gang boss that will defend the dungeon
        for sixty seconds.
    \item \emph{spawn\_random} (100) -- spawns a random combat minion that will defend the dungeon until its death.
\end{itemize}

\section{Buildings}

The building tool bar window, which can be seen in Figure~\ref{gui-build}, functions in the exactly same way as the spell tool bar does.
But buildings, unlike spells, are all placed in the same way -- we simply click on the button representing the building we want to build
or use one of the key bindings for actions \emph{SPELL/BUILD 1-4} and a model of the building will start to follow the mouse cursor. Once we
position our mouse cursor on an unobstructed place in the game world we can press the left mouse button to place the building.

\begin{figure}[H]
    \centering
    \includegraphics[width=\textwidth]{../img/tool-build.png}
    \caption{Window that allows the player to place buildings.}
    \label{gui-build}
\end{figure}

Most of the buildings can be unlocked through the research window.\footnote{Those that are not are available to the player from the start of
the game.} Similarly to spells, a research node \emph{BUILDING NAME} unlocks a
building called \emph{building\_name}. The next list contains all of the buildings our players can build, along with their price
in gold noted next to their names in parentheses and their description.

\begin{itemize}
    \item \emph{wall} (300) -- can be placed to separate rooms and to protect the dungeon.
    \item \emph{mine} (400) -- spawns a single ogre miner, which can mine walls and gold deposits.
    \item \emph{slow\_trap} (400) -- halves the speed of enemies that step on it for five seconds with a cooldown period
        \footnote{By \emph{cooldown period} we refer to the amount of time between two applications of the trap's effects.}
        of thirty seconds.
    \item \emph{mana\_crystal} (500) -- increases maximum mana and mana regeneration of the player.
    \item \emph{gold\_vault} (500) -- used to store gold.
    \item \emph{light\_crystal} (500) -- used as a source of light.
    \item \emph{fortified\_wall} (600) -- a strong wall that can be used to protect the dungeon.
    \item \emph{teleport\_trap} (600) -- teleports enemies that step on it to a random spot on the map with a cooldown period
        of thirty seconds.
    \item \emph{damage\_trap} (600) -- damages enemies that step on it with a cooldown period of thirty seconds.
    \item \emph{kill\_trap} (600) -- kills enemies that step on it with a cooldown period of thirty seconds.
    \item \emph{freeze\_trap} (800) -- freezes enemies that step on it in place for fice seconds with a cooldown period of thirty seconds.
    \item \emph{barracks} (1000) -- spawns a single ogre warrior, which is a melee combat minion.
    \item \emph{ice\_tower} (1500) -- spawns a single ogre ice mage, which is a ranged combat minion that can cast freezing waves.
    \item \emph{light\_mana\_crystal} (1500) -- used as a source of light which increases maximum mana and mana regeneration of the
        player.
    \item \emph{thunder\_tower} (1750) -- spawns a single ogre thunder mage, which is a ranged combat minion that can cast lightning bolts.
    \item \emph{fire\_tower} (1800) -- spawns a single ogre fire mage, which is a ranged combat minion that can cast meteors.
    \item \emph{church} (2000) -- spawns a single ogre cleric, which is a range combat minion that can heal others.
    \item \emph{chaos\_tower} (5000) -- spawns a single ogre chaos mage, which is a ranged combat minion that can cast random spells.
\end{itemize}

Beside these buildings, there is one additional building that the player cannot build, but is built by the game whenever a new level is
generated -- the dungeon throne.

\section{Options menu}

If we look at the options menu window, which can be accessed from either the main menu or the ingame menu bar, we will be presented with a
few basic options, which can be seen in Figure~\ref{gui-options}. These
options include the ability to change the resolution and fullscreen status of the game. Both of these options are controlled by a list
with predefined choices which can be clicked on to change the values that are located below the lists.
Once we choose new values for these options, we can apply them by clicking on the \emph{APPLY} button.

Below these graphical options, we can see pairs of buttons and labels. Each of these labels represents an action in the game and its
corresponding button the key that is assigned to action. To change the key binding, we can click on the button with our left mouse button
and then press a new key, which will be newly bound to the action. Note that this key binding change is only temporary and the key bindings
will reset when we restart the game. To save our new key bindings, we can use the \emph{APPLY} button, which will ensure that these new
key bindings will persist between games.

The \emph{SPELL/BUILD 1-4} actions correspond to the four buttons used to cast spells and buildings and their effect depends on the
currently selected tool bar. \emph{NEXT SPELL/BUILD} and \emph{PREV SPELL/BUILD} move the spell or building selection to the right and
to the left, respectively.

The \emph{SPELL TAB}, \emph{BUILD TAB} and \emph{MENU TAB} actions change the current tool bar between the spell selection, building
selection and mini menu. \emph{RESET CAMERA} returns the game's view back to the center of the map. Lastly, \emph{QUICK SAVE} and
\emph{QUICK LOAD} are used for fast and simple saving or loading of the save file \texttt{saves/quick\_save.lua}.

\begin{figure}[H]
    \centering
    \includegraphics[width=\textwidth]{../img/gui-options.png}
    \caption{Options menu which contains basic graphics options and key binding.}
    \label{gui-options}
\end{figure}

Besides these rebindable actions, the game contains three actions that are used mainly for mod prototyping and testing and cannot be rebound.
If we press the \emph{Grave} key\footnote{On starndard english, american and czech keyboards, this key is generally located above the
\emph{TAB} key and below the \emph{ESC} key and is used to input a semicolon or a tilde.} we will bring up the game's development console,
which accepts Lua code and can interpret this code with the \emph{EXECUTE} button. If we hold the \emph{Shift} key pressed while we click
on the \emph{Grave} key, we bring up the entity creator window that lets us to place all entities in the game without any cost. Lastly,
the \emph{0} key on the numeric pad toggles between fixed camera mode and free camera mode.


\chapter*{Conclusion and Future Work}
\addcontentsline{toc}{chapter}{Conclusion and Future Work}


%%% Bibliography
\include{bibliography}

%%% Attachments to the bachelor thesis, if any. Each attachment must be
%%% referred to at least once from the text of the thesis. Attachments
%%% are numbered.
%%%
%%% The printed version should preferably contain attachments, which can be
%%% read (additional tables and charts, supplementary text, examples of
%%% program output, etc.). The electronic version is more suited for attachments
%%% which will likely be used in an electronic form rather than read (program
%%% source code, data files, interactive charts, etc.). Electronic attachments
%%% should be uploaded to SIS and optionally also included in the thesis on a~CD/DVD.
\chapwithtoc{Attachments}

Attachments, which can be found on the attached DVD contain:

\begin{enumerate}[label=\textbf{\Alph*.}]
    \item Implementation of the game, which we can find in directory \texttt{tdt-project} and which contains source code, project files,
        compiled libraries and resources the game uses. It contains the following files and directories:
        \begin{itemize}
            \item Directory \texttt{bin} in which the Visual Studio solution generates the output of the compilation.
            \item Directory \texttt{deps}, which contains the Visual C++ Redistributable that is needed by the game.
            \item Directory \texttt{lib}, which contains \texttt{.dll} and \texttt{.lib} files of Ogre3D, CEGUI and Lua as well as their
                header files. These libraries were compiled from the official source code and were  included because there are no official
                precompiled packages of these libraries for Visual Studio 2015.
            \item Directory \texttt{res}, which contains the directory \texttt{ogre-configs} -- the contents of which need to be placed
                in the same directory as the game's executable file -- and the directory \texttt{resources}, which needs to be placed
                in the same directory as the game's executable file.
            \item Directory \texttt{src}, which contains the source code of our game.
            \item Visual Studio 2015 solution files, namely \texttt{tdt-project.sln} which represents the solution,
                \texttt{tdt-project.vcxproj} which represents the project and \texttt{tdt-project.vcxproj.filters} which represents
                the source code hierarchy in the project.
        \end{itemize}
    \item Documentation of the game and of the modding API, which we can find in directory \texttt{docs}. It contains the following files
        and directories:
        \begin{itemize}
            \item File \texttt{components.txt}, which contains a list of all components and information on how to include these
                in a table that defines an entity.
            \item Directory \texttt{lua-api}, which contains files that represent different modules of the Lua API.
            \item Directory \texttt{engine-doxy}, which contains HTML and PDF documentation of the \cpp engine generated by
                Doxygen~\cite{doxy}.
        \end{itemize}
    \item Compiled version of the game, which we can find in directory \texttt{tdt-game}. It contains the game's executable file
        called \texttt{tdt.exe} and all files the game needs in order to run -- resources, configurational files, scripts and libraries.
    \item Electronic version of this thesis.
\end{enumerate}

\openright
\end{document}
